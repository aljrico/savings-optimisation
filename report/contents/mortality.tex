\section{Optimisation of Pooled Funds Investment Strategies}

So far, we have managed to test two different strategies. We have analyzed their return and risk in a simulated scenario, within the context of a pension plan. In order to assess the risk of the strategies, so far we have been measuring the market risk of the inestment, by computing the \emph{Expected Shortfall}. However, in real-scenario pension plans, there are plenty of other risks that should be taken into consideration. One important risk, not rarely underestimated, is the so-called \textbf{Longevity Risk}. Longevity risk is the risk of retirees that will live longer than expected and will thus exhaust all their savings. This risk might doom some individuals to utter poverty or to burden relatives.

Recently, two worldwide phenomena ought to be highlighted. The collapse in low-risk assets returns as government bonds or blue chip stocks. And the observed demographic transition~\textcite{b:demographic, a:bongaarts-human}, in which both birth rates and death rates are plumbing down; increasing the life expectancy of elder individuals. The combination of these two factors is leading to an increase in longevity risk that the pension plans providers are facing, rising pension premiums and stagnating disposable incomes by savers and pushing them to work longer years before retirement.

As a response to this challenging, the work of~\cite{a:donnelly-transparency} and \cite{a:brautigam-pool} suggested a different approach to face longevity risk, the concept of \textbf{Pooled Funds}.

Pooled Funds are funds formed by many different individual savers that aggregate their savings together. The main characteristic of Pooled Funds is the fact that it takes into account the survival rate of the savers. Sadly, not all investors live long enough to cash back the profits of their investments. Pooled Funds make a strength out of this and proportionally redistributes the invested money of the deceased among the rest of the investors. Alongside other advantages, pooled funds benefit from economies of scale, cheaper diversification and a more efficient management of longevity risk.

In this section we will study the application of both CPPI and Alternative schemes that we have developed in previous sections under the framework of a simplified pooled fund.

\subsection{Simulating the Pooled Fund}

In order to simulate the pooled fund, we will construct a simple scenario where many investors of the same age start investing at the same time. We will take real death probabilities at each age, and we will simulate the death of some of the savers.

When savers die, some proportion $w$ of their saved money stays in the pool, benefiting the survivors. The rest is extracted from the pool, to their family or inheritors. In order to simulate the probability of death for each individual, we took the empirical measures of death probability for every age from the Spanish Government~\cite{o:mortality-table}. This way, we can assume that from a starting number of persons $n$ of the same age, and thus with the same death probability $p$, the number of persons that would die $X$ before the next year should follow a \emph{Binomial Distribution}. Thus the probability of $k$ deaths is:

\begin{align}
  Pr(X = k) = \frac{n!}{k!(n-k)!} p^k (1-p)^{n-k} \emph{.}
\end{align}

Now we can now make use of the algorithm described in \cite{a:schmeiser-binomial} to generate random numbers $k$ that follow a Binomial Distribution, and assume those $k$ to be the simulated deaths for the following year. Once we have $k$, we can compute how much proportional wealth $M$ has been given to every investor of the common pool by:

\begin{align}
  M = \frac{k}{n}w \emph{.}
\end{align}

The point of constructing the Pooled Fund in such a way, is that we are not altering the core principles of the assumptions taken building the Alternative Model. The relative proportion $\pi$ of wealth to be allocated in risky assets remains untouched, we are just altering the $x = X(t)$ present in  Equation~\ref{eq:maxE}. Resulting in

\begin{align}
	\max_{\pi}\EX\giventhat{u_{X(T)}}{x(1 + M)} \emph{.}
\end{align}

And thus the wealth of the investor will behave as

\begin{align}
	dX = \alpha \pi(t)X(t)dt + \sigma \pi(t)X(t)dW(t) + dC(t) + dM(t)X(t)w \textit{.}
\end{align}

In the snippet in Appendix~\ref{ap:mort-code} we show the necessary R code to perform this simulation for both CPPI and Alternative schemes.

\subsection{Results}

Now we are going to replicate the measures taken in the previous section but with the added conditioni of mortality and see whether or not the differences spotted between the CPPI and the Alternative strategies remain similar.

Firstly, we plot together the frequency histograms for the final wealth distribution for
both models.

\begin{figure}[h]
    \centering
    \includegraphics[scale=0.75]{./images/mort_final_wealth.png}
    \caption{Results of $1,000,000$ simulations for the both models with mortality. Final wealth obtained for every simulation. The blue one is the CPPI, and the red one is the Alternative. With $\pi = 0.5$, $\alpha = 0.343$, $\sigma = 0.1544$, $a = 10$, $T = 60$, $A = 0.5$ and therefore $ES/K = -3.3$.}
    \label{fig:mort_fw}
\end{figure}


Looking at Figure \ref{fig:mort_fw} we might note that the results are quite similar. Nevertheless, we can spot some differences in the shape of the non-central zones of the figure, where the Alternative Strategy seems to show less dispersion.

With a quick glimpse at Figure \ref{fig:mort_loss} we can already notice the great difference in the loss distribution of both schemes. It is clear that the CPPI is far more risky than the Alternative.

\begin{figure}[h]
    \centering
    \includegraphics[scale=0.75]{./images/loss_both_mort.png}
    \caption{Results of $1,000,000$ simulations for the both models with mortality. Losses obtained for every simulation. The blue one is the CPPI, and the red one is the Alternative. With $\pi = 0.5$, $\alpha = 0.343$, $\sigma = 0.1544$, $a = 10$, $T = 60$, $A = 0.5$ and therefore $ES/K = -3.3$.}
    \label{fig:mort_loss}
\end{figure}

Since the conclusions extracted from just one value of $\pi$ may be misleading, let us iterate these simulations for many values of $\pi$ and $K$, using Formula \ref{eq:kes} and see the results of Table \ref{tab:cppi_alt_mort}.

% \begin{table}[h]
% \centering
% \caption{Results of $100,000$ simulations of both CPPI and Alternative models in a Pooled Fund, using different risk levels, with $\alpha = 0.343$, $\sigma = 0.1544$, $a = 10$, $T = 60$, $A = 0.5$ and therefore $ES/K = -3.3$.}
% \label{tab:cppi_alt_mort2}
% \begin{tabular}{ccccccc}
% \textbf{$\pi$} & \textbf{ES } & \textbf{$K$} & \textbf{CPPI ret} & \textbf{Alt ret} & \textbf{diff}  & \textbf{equiv $\pi$}\\
% 0.1  & -                & -                & -               & -                  & -                & -                   \\
% 0.2  & -                & -                & -               & -                  & -                & -                   \\
% 0.3  & -                & -                & -               & -                  & -                & -                   \\
% 0.4  & -   & - & - & - & - & - \\
% 0.5  & -27.70 & 90.18  & 3.45 & 2.80    & -0.64 & 0             \\
% 0.6  & -98.78 & 321.52  & 3.80 & 4.96  & 1.16  & 2.37                \\
% 0.7  & -121.94 & 396.91  & 4.13 & 5.66  & 1.52 & 1.28              \\
% 0.8  & -143.56 & 467.30  & 4.74  & 6.03   & 1.29  & 3.01                   \\
% 0.9  & -168.69 & 549.08  & 4.78  & 6.48   & 1.70 & *                    \\
% 1 & -248.76 & 809.72  & 5.24  & 7.82  & 2.57 & *
% \end{tabular}
% \end{table}


\begin{table}[h]
\centering
\caption{Results of $100,000$ simulations of both CPPI and Alternative models in a Pooled Fund, using different risk levels, with $\alpha = 0.343$, $\sigma = 0.1544$, $a = 10$, $T = 60$, $A = 0.5$.}
\label{tab:cppi_alt_mort}
\begin{tabular}{ccccccc}
\textbf{$\pi$} & \textbf{ES } & \textbf{$K$} & \textbf{CPPI ret} & \textbf{Alt ret} & \textbf{diff}  & \textbf{equiv $\pi$}\\
10  & 140.32  & 18.11 & 1.84 & 1.97 & 0.13 & 0.15 \\
20  & 88.75  & 88.41 & 2.13 & 2.79 & 0.65 & 0.34 \\
30  & 58.50  & 96.45 & 2.66 & 2.80 & 0.13 & 0.32 \\
40  & 20.85  & 165.75  & 3.13 & 3.59 & 0.46 & 0.49 \\
50  & -28.11 & 232.00  & 3.46 & 4.17 & 0.70 & 0.64 \\
60  & -73.18 & 294.04 & 3.83 & 4.67 & 0.83 & 1.06 \\
70  & -118.58 & 381.23 & 4.24 & 5.40 & 1.15 & 1.63 \\
80  & -153.69 & 424.03 & 4.77 & 5.85  & 1.07 & *                 \\
90  & -223.58 & 524.09 & 4.88 & 6.77 & 1.88 & *                 \\
100 & -261.12   & 618.38 & 5.16 & 6.79 & 1.62 & *

\end{tabular}
\end{table}


The Pooled Fund has the effect of the wealth of all savers to increase dramatically. In fact, for lower risk levels as for $\pi < 0.5$, the final wealth of all simulations is so huge that the Expected Shortfall happenes to result positive and utterly large, instead of negative. So in order for the Alternative Strategy to equal that Expected Shortfall, the Formula \ref{eq:the_formula} leads to so hugely negative value of $K$ that the $\pi$ comes negative. Implying a short position in the market. Akwnoledging the non-sense that could derive from that decision, we have decided to cast out those simulations and simply state that this equivalence between the Expected Shortfall of both strategies is not valid for positive values of the Expected Shortfall.

As in the previous section, equivalent values of $\pi$ have been working as a succint metric to compare the performance of this two strategies, taking into account both return and risk. Hence, as long as the equivalent $\pi$ remains above than the initial $\pi$, we can say that the Alternative Strategy is outperforming
the CPPI. Which is clear for most of the values of initial $\pi$ and a fixed value for $A$.

Additionally, we find interesting to show the iterated simulation of Table \ref{tab:cppi_alt_mort} for many values of $A$ .

\begin{figure}[h]
    \centering
    \includegraphics[scale=0.75]{./images/pi-A_mort_2.png}
    \caption{Plot that shows the relation between equivalent $\pi$ and $A$, for many different initial $\pi$. From the results of $100,000$ simulations for each curve, with $\alpha = 0.343$, $\sigma = 0.1544$, $a = 10$, $T = 60$.}
    \label{fig:pi-a_mort}
\end{figure}

The main difference we can note between Figure \ref{fig:pi-A_all-pis} and Figure \ref{fig:pi-a_mort} is that out of a Pooled Fund, the decision of whether the CPPI strategy or the Alternative are more worth it, in terms of risk, has to do with the value of $A$ alone, not with $\pi$. On the contrary, within the context of a Pooled Fund, we can see that the Alternative model shines over the CPPI for higher values of $\pi$. In fact, for $\pi >= 0.7$ the Alternative strategy outperforms the CPPI so much that it simply does not exist an equivalent value $\pi_b$ for the CPPI to equal the return of the Alternative. This result is of great importance, specially considiering that both strategies incurr in the same risk level.

Despite that, there are some scenarios in a Pooled Fund where the CPPI outperforms the Alternative scheme. As before, we can see that this happens for huge levels of $A$, where $A$ is interpretd as the inverse of the risk aversion. Nonetheless, in Figure \ref{fig:pi-A_all-pis} we could clearly spot a downhill behaviour of the plot; which is not the case in Figure \ref{fig:pi-a_mort}. This leads us to highlight a sweetspot on $1.0 < A < 1.5$ where the Alternative strategy shows the comparatively better performance.





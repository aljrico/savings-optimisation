\section{Pooled Funds and Mortality}

One important risk, not rarely underestimated, is the so-called \textbf{Longevity Risk}. Longevity risk is the risk of retirees that will live longer than expected and will thus exhaust all their savings. This risk might doom some individuals to utter poverty or to burden relatives.

Recently, two worldwide phenomena ought to be highlighted. The collapse in low-risk assets returns as government bonds or blue chip stocks. And the observed demographic transition~\textcite{b:handbook-natural, a:bongaarts-human}, in which both birth rates and death rates are plumbing down; increasing the life expectancy of elder individuals. The combination of these two factors is leading to an increase in longevity risk that the pension plans providers are facing, rising pension premiums and stagnating disposable incomes by savers and pushing them to work longer years before retirement.

As a response to this challenging, the work of~\cite{a:donnelly-transparency} and \cite{a:brautigam-pool} suggested a different approach to face longevity risk,  the concept of \textbf{Pooled Funds}. 

Pooled Funds are funds formed by many different individual savers that aggregate their savings together. Alongside other advantages, pooled funds benefit from economies of scale, cheaper diversification and a more efficient management of longevity risk.

In this section we will study the application of both CPPI and Alternative schemes that we have developed in previous sections under the framework of a pooled fund. 


\subsection{Simulation}

In order to simulate the pooled fund, we will construct a simple scenario where many investors of the same age start investing at the same time. We will take real death probabilities at each age, and we will simulate the death of some of the savers. 

When savers die, some proportion $w$ their saved money stays in the pool, benefiting the survivors. The rest is extracted from the pool, to their family or inheritors.

\IncMargin{1em}
\begin{algorithm}[H]
  \caption{Simulated CPPI with mortality}
  \label{alg:cppi-mortality}
  \DontPrintSemicolon
  \SetKwFunction{LinearModel}{LinearModel}
  \SetKwFunction{GetStatistic}{GetStatistic}
  \SetKwFunction{Filter}{Filter}
  \SetKwFunction{SubSet}{SubSet}
  \SetKwFunction{Permute}{Permute}
    \KwData{Hurricane observational data $O$, with paired variables $X$, $Y$; classified by SST class ($C : \qty{low ,high}$), with $n$ and $m$ observations respectively}
    \KwResult{$p$-values defined under the null hypothesis}
    % initialization\;
    $O_{low}$ $\gets$ \SubSet($(x,y) \in O \mid c \equiv low$)\;
    $O_{high}$ $\gets$ \SubSet($(x,y) \in O \mid c \equiv high$)
    \tcp*{Notice that $O_{low} \cap O_{high} = O$}
    fit$_{low}$ $\gets$ \LinearModel($Y_{low} \sim X_{low}$)\;
    fit$_{high}$ $\gets$ \LinearModel($Y_{high} \sim X_{high}$)\;
    T $\gets$ \GetStatistic(fit$_{low}$, fit$_{high}$) \;
    count = 0 \;
    \For{$i \gets 1$ \textbf{to} $N$}{
      $O'$ $\gets$ \Permute($O$) \;
      $O_{low}\sast$ $\gets$ \SubSet($(x,y)_{i} \in O', \forall i \in [1, n]$ )\;
      $O_{high}\sast$ $\gets$ \SubSet($(x,y)_{i} \in O', \forall i \in [n+1, n+m]$ )
      \tcp*{$O_{low}\sast \cap O_{high}\sast = O'$}
      fit$_{low}\sast$ $\gets$ \LinearModel($Y_{low}\sast \sim X_{low}\sast$)\;
    fit$_{high}\sast$ $\gets$ \LinearModel($Y_{high}\sast \sim X_{high}\sast$)\;
      T$\sast$ $\gets$ \GetStatistic(fit$_{low}\sast$, fit$_{high}\sast$) \;
      \If{$\text{T}\sast > \text{T}$}{
        count $\gets$ count + 1
      }
    }
    \Return{$p$-value $\gets$ count / n.sim} \;
\end{algorithm}
\DecMargin{1em}
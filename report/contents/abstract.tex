\cleardoubleevenemptypage
\thispagestyle{empty}
\phantomsection
\addcontentsline{toc}{section}{Abstract}
\begin{abstract}

In classical savings theory it is stated that the optimal way to ensure a risk level is to set a constant proportion of risky assets at all time. This way, most savings strategies try to optimize their results based on a fixed risk aversion profile initially settled by the investor. Usually, funds arranging pension plans manage their fund as the result of the sum of many individually isolated investors, with no consideration upon the opportunities present in more collectively-managed schemes.

Throughout this work we will replicate the results of an alternative optimal strategy that changes proportion invested in risky assets along time, instead of setting an initial constan proportion; and we will confirm that this kind of strategy does not impede to set a fixed risk aversion profile. Moreover, we will incorporate the concept of Pooled Funds to those schemes and compare their performances using standard risk measures.

Finally, we will study the risk profile of all previously developed strategies on the article using Tail Distribution Modelling and Extreme Value Analysis in order to contrast the particularities of location and scale free risk measures.

\end{abstract}

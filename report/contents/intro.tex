%-----------------------------------------------------------------
%	INTRODUCTION
%	!TEX root = ./../main.tex
%-----------------------------------------------------------------
\section{Introduction}

The reason for saving is to utilise the present wealth of a saver in order to build a retirement plan that secures a safe stream of capital for when it may be needed. This future condition need can be thought as deterministic, as we usually think in a typical pension plan, when there is a fixed time scheme when the saver starts collecting his money and when it ends; or it can be subject to some non-deterministic eventuality, as in most insurance plans. This kind of investment is, thus, characterised by an initial period in which the investor is saving money followed by a period of consumption, once the investor satisfies that future condition.

Hence, every savings strategy should aim at maximising that final capital whilst \textit{securing} it. In general, we could say that there is a trade-off between that maximisation of capital (return) and the degree of its security (risk). The balance between these two magnitudes is what savings strategies try to optimise.

It is obvious that every saver would like to maximise the return of his money. But if we accept that this can only be accomplished at the expense of more risk, the investor has to decide which degree of risk is she able to tolerate, setting a risk limit. This decision upon the exposure to risk is what defines the \textit{risk aversion} profile of every investor.

Most savings strategies are measured setting a fixed risk limit provided by a risk aversion profile, thus maximising the returns that can be extracted once that risk limit is provided. Thus, the great interest canalized on testing and analysing the results of different savings strategies using the return and the risk as measures of performance.

% Throughout this work, we will make use of the risk adjusted ad hoc performance measure, recently developed in~\cite{a:guillen-performance, a:guillen-guarantee}, in order to evaluate our models.

Throughout this project we will compare the results obtained from many different savings strategies, and we will compare their risk and return. The set up for all of them will be the same. A simulated investor will save up a yearly fixed amount of money from the age of 30 years and, after that, she will consume the same amount for the next 30 years. Part of the saved money will be invested in risky assets whilst the rest will be invested in risk-free assets. The main question to be addressed is going to be the optimal proportion of the investment exposed to risky assets, and whether the fund is managing the investors isolated or collectively.

Firstly, we will study the results obtained from~\cite{a:guillen-optimisation} and compare the two strategies involved in their study: The benchmark and what we call the \emph{Alternative} strategy.

In the case that we use as a benchmark, called \textit{Constant Proportion Portfolio Insurance} (CPPI), introduced by \cite{a:perold-constant}, a constant proportion of the wealth to be invested is allocated in risky assets, whereas the rest is allocated in non-risky assets. This means that the same proportion of the investment will be invested in the risky market, year over year.

The Alternative scheme, developed in \cite{a:guillen-optimisation}, consists of a different approach regarding that proportion of risky investment. Instead of a time-constant proportion, this model suggests a variable proportion that follows a formula that takes into account the present wealth of the investor. We will explain this formula in the following chapters.

Later on, we will include the concept of a \emph{Pooled Fund}, in which the savers are not considered in isolation, but parts of a whole Fund. The main characteristic of considering the saver within a common Pooled Fund is the concept of mortality. There is always a risk that the saver does not live up to retire and collect back the fruit of her savings. Taking that risk into account, we will show how the wealth of those unlucky deceased investors can be distributed among the survivors. Moreover, we will test both previously described strategies in the context of a Pooled Fund and analyse their performance.

Throughout the majority of this work, we will make use of the risk adjusted ad hoc performance measure, recently developed in~\cite{a:guillen-performance, a:guillen-guarantee}, in order to evaluate our models. Yet in the final section we will develop and explain some other measures of Risk that are more sensible to worst-case scenarios and larger losses, becoming another useful tool to assess risk and performance of Investment and Savings Strategies.

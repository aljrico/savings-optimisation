\section{Conclusions}

As a result of this work we have given a bit of context and explanation to standard risk measures and utilised them to revise and expand the results obtained from \cite{a:guillen-optimisation}. It has been shown that for more values for $\pi$ and $A$ the same conclusions can be extracted: If a saver is capable of assuming up to $K$ losses, then the optimal strategy is to invest $A$ times the accumulated wealth in risky assets, where $A$ is defined as the inverse of the risk aversion.

Additionally, following the logic provided by the comparison of the benchmark and the proposed strategy, the same analysis has been applied within the context of a Pooled Fund, that takes into account mortality. First, the purpose has been to present and simulate the results obtained of utilising such scheme, and we have seen that the computed returns presented in Table~\ref{tab:cppi_alt_mort} corresponding to the Pooled Fund are considerably higher than the presented in Table~\ref{tab:cppi_alt} corresponding to the simpler approach. These results highlight the potential profits that the industry could gain from using such schemes in Pension Plans, and the necessary academic interest in studying them.

Then, we aimed to test whether the same conclusions and insights gotten from~\cite{a:guillen-optimisation} can be extrapolated when adding Mortality and a Pooled Fund. The results presented in Figure~\ref{fig:es-es_mort_old} indicates that whereas some of the assumptions considered in~\cite{a:guillen-optimisation} in order to develop Equation~\ref{eq:kes} are clearly correct (see Figure~\ref{fig:es-es_pi2}), they might not be holding true anymore within the context of Pooled Funds. Additionally, results presented in Figure~\ref{fig:pi-a_mort} have shown us that there the same conclusion can be extracted without and within Pooled Funds: If a saver is capable of assuming up to $K$ losses, then the optimal strategy is to invest $A$ times the accumulated wealth in risky assets, where $A$ is defined as the inverse of the risk aversion. This stands of even greater importance for values of $A$ closer to $1.5$. And, in many cases, the performance of the alternative strategy dwarfs the returns obtained by the CPPI, as much as concluding that there is no achievable level of risk that lets the CPPI equal the higher results obtained by the alternative strategy.

Moreover, we have added to these previous analysis the usage of a new kind of risk measure based on the Extreme Value Analysis of the tails of the obtained distributions, providing a location and scale free risk measure, especially in situations where location-dependent measures are less reliable. The purpose has been to highlight the main differences and limitations of different risk measures and stress the necessity of adding location and scale free risk measures, as the Extreme Value Index, to the standard toolkit of risk analysis when optimising financial investments and savings strategies. When applied to the strategies of interest, we have been able to spot some interesting insights, as the decrease of the thickness of the tails obtained by the strategies where the Alternative Scheme is used. Indicating that even for the same values of Expected Shortfall, we can find that one strategy shows less risk than the other, helping the investor or institution contrast the risk level of different strategies.

An immediate consequence of adding mortality to the scheme of any strategy is that the investor the survives the long run manages to largely increase their wealth. This shrinkes the risk of losses dramatically. This kind of risk has to be necessarily different in nature from the risk just derived from the dispersion of the expected return. From the results extracted from Figure~\ref{fig:evi-es} we might deduce that this condition is fairly better reflected by the Extreme Value Index rather than the Expected Shortfall, proving that the EVI can function as a location and scale free risk measure.

We conclude that extreme caution ought to be taken when making conclusions out of apparently less risky scenarios; and that a Location and Scale free risk measure could give more insight in such situations. Furthermore, pension plans providers could benefit greatly from using of both the Alternative Model and the context of Pooled Funds, separated or together, for most risk profiles. With the added simplicity of using maximum allowed loss $K$ as a metric for settling the risk profile of savers, instead of more abstract ones as "risk aversion".

Further research could be of extreme interest on the Pooled Fund scheme. The structure presented in this work is utterly simplistic and could be extended. Major improvements could reside in studying more levels for the percentage returned to the pool, when an investor dies (in this work we have always considered a $100\%$, even acknowledging that that could be unrealistic) or adding age variability to the simulated investors.
